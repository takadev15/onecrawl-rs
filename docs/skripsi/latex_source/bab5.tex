%!TEX root = ./template-skripsi.tex
%-------------------------------------------------------------------------------
%                            	BAB IV
%               		KESIMPULAN DAN SARAN
%-------------------------------------------------------------------------------

\chapter{KESIMPULAN DAN SARAN}

\section{Kesimpulan}
Berdasarkan hasil implementasi dan pengujian fitur sistem informasi yang telah dirancang, maka diperoleh kesimpulan sebagai berikut:

\begin{enumerate}
  \item \emph{Crawler} berhasil mengumpulkan data dari halaman web yang \emph{domain}-nya telah di definiskan dalam \emph{origin url}.
  \item Dari perbandingan hasil proses \emph{crawling} bahasa pemograman berabstraksi rendah lebih cocok untuk digunakan dalam \emph{high-intensity application} seperti \emph{web crawler} ini.
  \item Migrasi \emph{database} dari berbasis \emph{SQL} menuju berbasis \emph{MonggoDB} berhasil dan data yang tersimpan konsisten.
  \item \emph{Crawler} yang dirancang menggunakan metode \emph{multi-threading} berhasil mengumpulkan jumlah halaman web lebih banyak daripada \emph{crawler} sebelumnya.
  \item Algoritma \emph{breadth-first search} termodifikasi dalam skripsi ini belum cukup untuk meningkatkan akurasi proses \emph{crawling} terhadap halaman web yang didefiniskan oleh \emph{top-level domain}.
  \item Penggunaan \emph{resource} lebih banyak berada di \emph{scouter service} bila dibandingkan dengan \emph{parser service}.
\end{enumerate}

\section{Saran}
Adapun saran untuk penelitian selanjutnya adalah:
\begin{enumerate} 
  \item Melanjutkan penelitian dalam eksplorasi algoritma \emph{crawler} lain untuk meningkatkan akurasi pengumpulan halaman web yang telah didefinisikan oleh \emph{top-level domain}.
  \item Melanjutkan penelitian dalam eksplorasi algoritma \emph{information retrieval} yang mengakomodasi lebih banyak jenis \emph{website}.
  \item Eksplorasi penggunaan \emph{filesystem} sebagai platform untuk menyimpan data hasil \emph{crawling} untuk mengakomodasi struktur halaman web yang berbeda-berbeda.
  \item Eksplorasi implementasi \emph{distributed crawler} dengan menggunakan skema \emph{multi-threading} dengan bahasa pemograman berabstraksi rendah seperti \emph{Rust}, \emph{C/C++}, atau \emph{Zig}.
\end{enumerate}


% Baris ini digunakan untuk membantu dalam melakukan sitasi
% Karena diapit dengan comment, maka baris ini akan diabaikan
% oleh compiler LaTeX.
\begin{comment}
\bibliography{daftar-pustaka}
\end{comment}

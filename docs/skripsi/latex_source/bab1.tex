
%!TEX root = ./template-skripsi.tex
%-------------------------------------------------------------------------------
% 								BAB I
% 							LATAR BELAKANG
%-------------------------------------------------------------------------------

\chapter{PENDAHULUAN}

\section{Latar Belakang Masalah}

% Intro to Search engine architecture and web crawler
\emph{Search engine} merupakan sebuah program yang digunakan untuk menjelajahi dan mencari informasi dari web \citep{seymour2011history}. Terdapat beberapa komponen yang membangun arsitektur \emph{Search engine} seperti \emph{Web crawler}, \emph{Page rank}, dan \emph{indexer} \citep{brin1998anatomy}. Dalam proses pencarian web yang dilakukan \emph{Search engine} tahap pertama yang di lakukan adalah \emph{Web crawler} menjelajahi dan mengekstraksi data-data dari list \emph{url} lalu menyimpan data tersebut dan data lain yang terkait ke dalam database \citep{brin1998anatomy}. Data yang disimpan akan di-index, diberikan skor dan di urutkan melalui algoritma \emph{pagerank} \citep{brin1998anatomy}.

% Analysis for past web crawler, algorithm, architecture, and tech stack
\emph{Web Crawler} merupakan komponen penting dalam pembuatan arsitektur \emph{Search engine} secara keseluruhan. Penelitian sebelumnya yang telah dilakukan oleh \emph{Lazuardy Khatulisitwa} telah berhasil mengimplementasikan \emph{Web crawler} kedalam arsitektur \emph{Search engine} yang berjalan \citep{lazuardithesis}. \emph{Web crawler} tersebut mengimplentasikan algoritma \emph{Breadth First Search} dengan modifikasi algoritma \emph{similiarity based} untuk meningkatkan akurasi dari proses \emph{crawling} dan pengambilan data dari  suatu halaman \citep{lazuardithesis}. Algoritma \emph{Modified Similarity-Based} yang digunakan oleh \emph{Fathan} untuk memperbaiki akurasi dari \emph{Breadth First Search} memanfaatkan konsep penyimpanan \emph{queue} dalam melakukan proses \emph{crawling} \citep{fathanthesis} . Dalam proses tersebut \emph{crawler} akan menyimpan 2 jenis \emph{queue} yaitu, \emph{hot queue} untuk menyimpan \emph{url} yang mengandung kata \emph{anchor} sedangkan \emph{url queue} digunakan untuk menyimpan \emph{url} lain \citep{cho1998efficient}. Proses ini dapat membantu \emph{crawler} untuk mengunjungi dan melakukan \emph{crawling} ke dalam \emph{page} yang terdapat di \emph{hot queue} terlebih dahulu, bila \emph{page} yang berkaitan dengan kata \emph{anchor} di kunjungi terlebih dahulu maka \emph{child page}-nya kemungkinan besar akan memiliki konten yang berkaitan dengan kata \emph{anchor} tersebut \citep{cho1998efficient}. 

Arsitektur dari \emph{crawler} yang di kembangkan oleh Lazuardi, menggunakan \emph{python} sebagai bahasa pemograman dan \emph{library} pendukung yang digunakan adalah \emph{beautifulsoup4} untuk melakukan \emph{parsing} dari halaman \emph{website}, \emph{request} untuk mengirimkan request kepada halaman \emph{website} yang ingin di ambil data-nya, dan \emph{regex} untuk melakukan pencocokan kata - kata yang telah di dapat dengan \emph{keyword} yang sudah di tentukan \citep{lazuardithesis}. Dari hasil penelitian \emph{lazuardi} terdapat beberapa saran peningkatan yang tercatat, dimana salah satunya terkait dengan meningkatkan kinerja dan peforma dari \emph{web crawler} agar memiliki penggunaan \emph{RAM} yang lebih kecil dan mencapai kinerja yang maksimal \citep{lazuardithesis}.

Kinerja atau performa dari \emph{web crawler} dapat dinilai dari waktu yang diperlukan oleh \emph{web crawler} dalam menjalankan fungsi-fungsi nya. Semakin cepat waktu eksekusi dari fungsi dalam \emph{web crawler}, semakin cepat satu halaman di proses dan semakin banyak informasi yang disimpan. Informasi terkait waktu yang dilewati oleh \emph{crawler} per bagian dan bagian apa yang menghambat performa program, dapat dilakukan dengan \emph{profiling}. \emph{Profiling} adalah mekanisme dalam menghitung waktu per-pemanggilan fungsi atau bagian dari kode sehingga dapat ditentukan bagian mana dalam kode yang paling menghambat performa program. \emph{Profiling} yang dilakukan di \emph{crawler} milik lazuardy, menggunakan \emph{line profiler} sebagai alat pembantu. Target dari \emph{profiling} merupakan modul \emph{crawler} dalam file \emph{crawl.py}, fungsi, dan metode-metode lain yang dipanggil oleh modul tersebut. Profiling ini dilakukan selama 6 jam dan menggunakan 4 \emph{threads} dengan taret \emph{url} sejumlah 3.

\begin{table}[H]
  \caption{Hasil \emph{profiling} crawler lazuardy}
  \begin{center}
    \begin{tabular}{ |p{1cm}|p{5cm}|p{4cm}|p{3cm}| } \hline
      \multicolumn{4}{|c|}{Hasil Profiling} \\ \hline
      No.& Nama Operasi& \emph{Library}& Waktu Operasi Rata-Rata (ms) \\ \hline
      1& Pengunduhan halaman \emph{website}& \emph{request}& 1,000 ms \\ \hline
      2& Penyusunan \emph{language tree}& \emph{beautifulsoup4}& 280.5 ms \\ \hline
      3& Pengambilan data dari \emph{queue}& \emph{queue}& 160.1 ms \\ \hline
    \end{tabular}
  \end{center}
\end{table}

Dari hasil \emph{profiling} dapat dilihat bahwa \emph{bottleneck} terbesar adalah proses pemanggilan \emph{request method} terhadap \emph{page}, proses ini memakan waktu rata-rata 1,000 ms atau 1 sekon yaitu sekitar 27 persen dari keseluruhan waktu pemrosesan satu halaman \emph{website}, diikuti oleh proses pembuatan \emph{language tree} yang memakan waktu rata-rata 280.5 ms per pemanggilan, dan pengambilan data \emph{url} dari \emph{queue}. Walaupun arsitektur \emph{search engine} milik lazuardy sudah menerapkan \emph{multi-threading} dalam proses \emph{scraping} per halamannya, antara proses penguduhan halaman dan penyusunan \emph{language tree} masih berada di dalam satu \emph{threads of execution} yang berarti kedua proses itu berjalan secara berurutan, ini mengakibatkan perlambatan yang signifikan dalam proses penjelajahan \emph{crawler} pada satu website dengan domain yang sama. Perbaikan yang dapat dilakukan berdasarkan fakta tersebut adalah dengan menggunakan protokol dan metode pengunduhan halaman \emph{website} yang lebih baik, pemisahan proses pengunduhan dan pembangunan \emph{language tree} sehingga dapat berjalan secara bersamaan, dan menggunakan bahasa pemograman yang memiliki efisiensi peforma yang lebih baik dari arsitekur saat ini.

Salah satu metode untuk mempercepat jalannya \emph{search engine} adalah \emph{Multi-threading} \citep{multithreadedtextsearch}. Metode ini sudah pernah digunakan dalam \emph{search engine} sebelumnya, tetapi \emph{search engine} ini mencari data bukan ke \emph{web} tetapi pada kumpulan data teks atau dapat disebut dengan nama \emph{text search} \citep{multithreadedtextsearch}. Dari hasil penelitian tersebut ditemukan metode \emph{multi-threading} yang digunakan berhasil mencapai improvisasi yang sebelumnya membutuhkan waktu 16 menit dalam menjelajahi seluruh data teks menjadi 4 menit, yang berarti berhasil mencapai improviasi waktu eksekusi program sebesar 4x \citep{multithreadedtextsearch}. Dalam penelitian tersebut metode \emph{multi-threading} digunakan untuk memecah proses pengambilan data dari sumber data dan proses parsing dari data teks yang sudah di ambil \citep{multithreadedtextsearch}.

Dalam konteks \emph{search engine} untuk pencarian web penelitian yang dilakukan oleh \emph{Pramudita, Y.D et all} telah menunjukkan bahwa mekanisme \emph{multi-threading} dapat diimplementasi dengan benar \citep{Pramudita_2020}. Dalam penelitian tersebut tiap-tiap \emph{thread} menjalankan satu \emph{instance} dari \emph{crawler} nya itu sendiri, dan penelitian tersebut berhasil mencapai percepatan waktu \emph{crawling} selama 123 detik \citep{Pramudita_2020}.

Selanjutnya penelitian hanya akan melakukan improvisasi terhadap komponen \emph{web crawler} saja untuk membatasi area penelitian. Penelitian ini akan berusaha untuk meningkatkan performa, yang dimana merupakan jumlah halaman yang terkumpul pada waktu yang sudah definisikan. Berdasarkan hasil penelitian \emph{Pramudita, Y.D et all}, yang dimana menjalankan keseluruhan proses \emph{crawler} dalam satu thread \citep{Pramudita_2020}, maka penelitian ini akan berusaha untuk meningkatkan performa dengan memisahkan proses \emph{parsing} dalam \emph{crawler} dalam proses yang berbeda atau yang dapat disebut dengan metode \emph{multi-processing}. Selain itu penelitian ini juga akan berusaha untuk meningkatkan akurasi hasil proses \emph{crawling} dengan menggunakan algoritma \emph{breadht-first search} yang dimodifikasi dengan tujuan agar \emph{crawler} hanya menjelajahi domain yang telah ditentukan saja, sehingga diharapkan hasil proses \emph{crawling} hanya akan berisi halaman web yang diinginkan. Perbaikan lain yang akan dilakukan adalah dengan menggunakan bahasa pemograman dengan waktu eksekusi yang lebih cepat, yaitu \emph{rust} \citep{RustPerformance}. Keputusan ini didasari dari hasil pengujian bahasa pemograman \emph{rust} dalam proses dengan intensitas tinggi dan konteks \emph{low-level} \citep{RustPerformance}.


\section{Rumusan Masalah}
Berdasarkan uraian pada latar belakang yang diutarakan di atas, maka perumusan masalah pada penelitian ini adalah “Bagaimana cara meningkatkan performa jalan \emph{web crawler} dalam \emph{search engine} menggunakan metode \emph{multi-threading} dan \emph{multi-processing} dengan bahasa pemograman \emph{rust}?”

\section{Pembatasan Masalah}
Adapun batasan-batasan masalah yang digunakan agar lebih terarah dan sesuai dengan yang diharapkan serta terorganisasi dengan baik adalah:
\begin{enumerate}
  \item Modifikasi akan dilakukan terbatas pada modul \emph{crawling} berdasarkan arsitektur \emph{search engine} milik lazuardy.
  \item Komparasi \emph{search engine} dilakukan pada mesin komputer yang sama dengan menggunakan koneksi internet yang sama.
  \item Modifikasi tidak akan mengubah model data dalam \emph{database} \emph{search engine}.
\end{enumerate}

\section{Tujuan Penelitian}
\begin{enumerate}
  \item Meningkatkan peforma jalannya \emph{web crawler} dalam \emph{search engine}.
  \item Meningkatkan akurasi hasil proses \emph{web crawling}.
  \item Menguji penggunaan metode \emph{multi-processing} dalam \emph{web crawler}.
\end{enumerate}

\vspace{1.0cm}

\section{Manfaat Penelitian}
\begin{enumerate}
	\item Bagi penulis

  Menambah pengetahuan dalam perancangan program dengan abstraksi rendah dan performa tinggi serta menerapkannya dalam bentuk program \emph{web crawler}.
		
	\item Bagi Program Studi Ilmu Komputer

    Penelitian ini menjadi langkah awal perbaikan \emph{search engine} dengan pembuatan \emph{crawler} berperforma tinggi, dan dapat memberikan gambaran bagi seluruh mahasiswa khususnya bagi mahasiswa program studi Ilmu Komputer Universitas Negeri Jakarta tentang proses pembuatan desain perancangan \emph{crawler} dan aplikasi \emph{multi-threding} sebagai pendukung pada penelitian \emph{search engine}.
		
	\item Bagi Universitas Negeri Jakarta 
	 	
	Menjadi pertimbangan dan evaluasi akademik khususnya Program Studi Ilmu Komputer dalam penyusunan skripsi sehingga dapat meningkatkan kualitas akademik di program studi Ilmu Komputer Universitas Negeri Jakarta serta meningkatkan kualitas lulusannya.
	 			
\end{enumerate}


% Baris ini digunakan untuk membantu dalam melakukan sitasi
% Karena diapit dengan comment, maka baris ini akan diabaikan
% oleh compiler LaTeX.
\begin{comment}
\bibliography{daftar-pustaka}
\end{comment}
